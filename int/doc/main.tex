\documentclass[12pt]{article}

% Document Setup
% ==============

\author{Steffen Haug}
%% remove the temporary files (.bcf, .aux, ...) if
%% you change the language, because this invalidates
%% the cached babel settings.
\usepackage[nynorsk]{babel}

\usepackage{csquotes}
\usepackage[style=phys]{biblatex}
\usepackage{scalerel}
\usepackage{relsize}

\usepackage{bm, upgreek}

\usepackage{mathtools} % coloneqq
\usepackage{siunitx}
\usepackage{textcomp}

\usepackage{unicode-math}

% Get mathbb and mathcal back (unicode-math replaces them)
\DeclareMathAlphabet{\mathcal}{OMS}{cmsy}{m}{n}
\let\mathbb\relax % remove the definition by unicode-math
\DeclareMathAlphabet{\mathbb}{U}{msb}{m}{n}

\setmainfont{EB Garamond}
\setmonofont{Courier New}
\setmathfont[StylisticSet={7, 2, 10}]{Garamond-Math}


% Custom Commands
% ---------------
\newcommand{\Vect}[1]{{\mathbf{#1}}}
\let\vec\Vect

\newcommand{\Mat}[1]{\mathbf{\mathrm{#1}}}

\newcommand{\deldel}[2]{\frac{\partial #1}{\partial #2}}
\newcommand{\deldelN}[3]{\frac{\partial^{#3} #1}{\partial #2 ^{#3}}}
\newcommand{\dd}[2]{\frac{\mathrm d #1}{\mathrm d #2}}
\newcommand{\textdd}[2]{{\mathrm d #1}/{\mathrm d #2}}
\newcommand{\Tay}[3]{T_{#2}\left \{ #1 \right \}(#3)}
\newcommand{\LRSeq}[1]{\left\{ #1 \right\}}
\newcommand{\Seq}[1]{\big\{ #1 \big\}}

%% fourier transform
\newcommand{\FT}[2]{
    \frac{1}{\sqrt{2\pi}} \smashoperator{\int\limits_{#1=-\infty\mathstrut}^{#1=\infty\mathstrut}}
        #2 e^{-i\omega #1} \D #1
}

\newcommand{\Err}[1]{\mathrm{error}_{#1}}
\newcommand{\Diam}{\mathrm{diam}}
\newcommand{\D}{\mathrm{d}}

\newcommand{\R}{\mathbb{R}}
\newcommand{\N}{\mathbb{N}}
\newcommand{\C}{\mathbb{C}}

\newcommand{\Norm}[2][]{\left\lVert#2\right\rVert_{#1}}
\newcommand{\Jac}[1][]{\mathbf{J}_{#1}}
\newcommand{\Diag}[1]{\mathrm{diag} #1}


\newcommand{\Fig}[1]{\mbox{\scshape figure \ref{fig:#1}}}
\newcommand{\Tab}[1]{\mbox{\scshape table \ref{table:#1}}}

\DeclareMathOperator*{\Sub}{\Bigg\vert}

% smash bounding boxes. useful for limits etc.
% \def\mathclap#1{\text{\hbox to 0pt{\hss$\mathsurround=0pt#1$\hss}}}

% -- Text Formatting
\linespread{1.10}

% -- Custom Titles
\usepackage{titlesec}

\renewcommand\thesection{\Roman{section}}
\renewcommand\thesubsection{\normalfont\roman{subsection}}
\titleformat{name=\section,numberless}[block]{\Large\scshape}{}{0pt}{}
\titleformat{name=\section}[block]{\Large\scshape}{\thesection.}{5pt}{}

\titleformat{name=\subsection,numberless}[block]{\filcenter\large\scshape}{}{0pt}{}
\titleformat{name=\subsection}[block]{\filcenter\large\scshape}{\thesubsection.}{5pt}{}

\usepackage{moresize}
\usepackage{abstract}
\usepackage{appendix}

\renewcommand{\abstractnamefont}{\normalfont\scshape}
\renewcommand{\abstracttextfont}{\normalfont\small}

% -- Title Section
\usepackage{titling} % Customising the title section
\setlength{\droptitle}{-5\baselineskip} % Move the title up

% Document margins optimized for two-column layout; they
\usepackage[
    marginparwidth=35mm,
    right=15mm,
    left=15mm,
    head=15pt,
]{geometry}
\setlength{\columnsep}{5mm}
\usepackage{afterpage}
\usepackage{pagecolor}

% -- Headers and footers
\usepackage{fancyhdr}
\renewcommand{\headrulewidth}{0pt}
\pagestyle{fancy}
\fancyhead{}
\fancyhead[C]{MA2501}
\fancyfoot{}
\fancyfoot[C]{\thepage}


% clean up the "plain" style.
\fancypagestyle{plain}{\
  \fancyhf{}
  \renewcommand{\headrulewidth}{0pt}
}


\usepackage{float}
\usepackage{multicol}
\usepackage{graphics}

% -- Custom captions
\usepackage[hang,small,labelfont=bf,up,textfont=it,up]{caption}

% -- Compact lists
\usepackage{enumitem}
\setlist[itemize]{noitemsep}

% -- LISTINGS --
\usepackage{listings}
% Options for all coding environments
\lstset{
    basicstyle=\scriptsize\ttfamily
}
\lstset{
    frame=tb,
    framexleftmargin=2.5em,
    numbers=left,
    xleftmargin=2.5em,
    showstringspaces=false,
}

% The built-in python highlighter is for python2,
% so we need to modify it:
\newcommand\pythonstyle{\lstset{
language=Python,
morekeywords={self, yield, as, assert, pass},
}}

% python env.
\lstnewenvironment{python}[1][]
{
\pythonstyle
\lstset{#1}
}
{}

% inline pytohn env.
\newcommand\pythoninline[1]{{\pythonstyle\lstinline!#1!}}

\usepackage{xcolor}
\usepackage{tikz}
\usepackage{standalone}

\usepackage{caption}
\captionsetup{format = plain, font = footnotesize, labelfont = {sc}}

\usepackage{multicol}
\usepackage{marginnote}


\addbibresource{b.bib}

\definecolor{ThemeBG}{HTML}{C1DCBD}
\definecolor{ThemeFrame}{HTML}{000000}
%% backround on front-page
\usepackage[firstpage=true]{background}

\begin{document}
\begin{titlepage}
    \backgroundsetup{
        contents={}
    }
    \newgeometry{inner=0pt, outer=0pt}
    \pagecolor{ThemeBG!70}
    \afterpage{
        \nopagecolor
        \restoregeometry
    }
    \begin{center}
        \noindent\fcolorbox{ThemeFrame}{ThemeBG}{
            \parbox{0.5\textwidth}{
                \begin{center}
                    \vspace{1em}
                    {\large\fontspec{Gill Sans}
                    \addfontfeature{LetterSpace=20.0}STEFFEN HAUG}\\[2em]
                    {\Huge\it Numeriske Metodar}
                    \vspace{1.618em}
                \end{center}
            }
        }
        \vfill
        {\ttfamily Prosjekt 3}
    \end{center}
\end{titlepage}

\begin{multicols*}{2}

    \section{Kvadraturmetodar}


    \subsection{Adaptiv Simpson-kvadratur}
\begin{python}[caption={Adaptiv Simpson-kvadratur}]
def asm_quad(f, a, b, tol=1e-5):

    def S(a, b):
        return abs(b - a) / 6 \
             * (f(a) + 4 * f((a + b) / 2) + f(b))

    I_0 = S(a, b)
    c = (a + b) / 2

    I = S(a, c) + S(c, b)
    err = abs(I - I_0) / 15

    if err <= tol:
        return I + (I - I_0) / 15

    return asm_quad(f, a, c, tol=tol/2) \
         + asm_quad(f, c, b, tol=tol/2)
\end{python}


    \subsection{Romberg-kvadratur}

\begin{python}[caption={Adaptiv Simpson-kvadratur}]
def romberg(f, a, b, MAX_ITER=100, tol=1e-5):
    R = np.full(shape = (2, MAX_ITER),
                fill_value = np.nan)
    Rp, Rn = 0, 1

    h = b - a
    R[Rp, 0] = 0.5 * h * (f(a) + f(b))

    for n in range(1, MAX_ITER):
        h = h * 0.5
        L = np.linspace(a + h, b - h, 1 << n - 1)
        R[Rn, 0] = R[Rp, 0]/2 + h * np.sum(f(L))

        for k in range(1, n + 1):
            E = (R[Rn, k - 1] - R[Rp, k - 1]) \
              / ((1 << 2 * k) - 1)
            R[Rn, k] = R[Rn, k - 1] + E

        Rp, Rn = Rn, Rp

        if abs(E) < tol:
            break

    return R[Rp, n]
\end{python}





    \section{Simulasjon av fri, stiv lekam}
    Vi ønsker å simulere ein {\em fri, stiv lekam},
    med andre ord ein lekam som ikkje lar seg deformere,
    som roterer fritt i rommet frå ein gitt start-tilstand.

    Fri rotasjon betyr at netto påført dreiemoment er null.
    At lekamen ikkje lar seg deformere medfører at
    treighetsmomentet er konstant. (massen flyttar seg ikkje relativt
    til rotasjons-aksen)
    Dette betyr at normen til dreieimpulsen, og rotasjonsenergien er bevart. \cite{lien}
    Ingenting hindrar dreieimpulsen i å endre retning.

    \subsection{Definisjon av problemet}
    Differensiallikninga
    \cite[Namn på symbol er endra for å vår oppgåvetekst]{lien}
    \begin{equation}
        \Mat T \deldel{\vec \upomega}{t} + \vec \upomega \times \Mat T\vec \upomega = \Mat M
        \label{_euler}
    \end{equation}
    skildrar rotasjonen til eit stivt legeme.
    Her er $\Mat M$ påført dreiemoment,
    $\Mat T$ treighetsmoment, og $\vec \upomega$ vinkelfart.
    Per antagelse er $M = 0$. Innfør substitusjonen
    \begin{align*}
        \vec m &= \Mat T \vec \upomega \\
        \implies \deldel{\vec m}{t} = \Mat T \dd{\upomega}{t} \quad&\text{og} \quad
        \vec \upomega = \Mat T^{-1} \vec m
    \end{align*}
    sett inn i \eqref{_euler},
    trekk frå $\textdd{\vec m}{t}$ på begge sider,
    og snu kryss-produktet for å få like forteikn.
    Vi skriv $\dot f = \textdd{f}{t}$.
    \begin{equation}
        \dot \vec m = \vec m \times \Mat T^{-1} \vec m
        \label{euler}
        \tag{\ref{_euler}*}
    \end{equation}
    Bevaringslovene nevnt til å byrje med gir oss:
    \begin{align}
        \label{sph}
        \gamma &= \Norm{\vec m}^2 = \vec m(t) \cdot \vec m(t) \\
        \label{ell}
        E &= \frac{1}{2} \vec m(t) \cdot \Mat T^{-1} \vec m(t)
    \end{align}
    Der \eqref{sph} er ei sfære med radius  $\Norm{\vec m}$,
    og \eqref{ell} er ei ellipse.
    Sidan {\em begge} er konstantar er løysingane $\vec m$ begrensa
    til å sitje på skjæringa mellom flatene.

    %% $\mathscr J$
    Anta $\Mat T = \Diag(\Mat I_1, \Mat I_2, \Mat I_3)$.
    Ettersom $\Mat T$ er diagonal er
    $\Mat T^{-1} = \Diag(1/\Mat I_1, 1/\Mat I_2, 1/\Mat I_3)$
    Skriv $\vec m = (x \; y \; z)$, og skriv ut \eqref{euler}
    på komponentform:
    \begin{align*}
        \underset{\dot \vec m}{\begin{pmatrix}
            \dot x(t) \\ \dot y(t) \\ \dot z(t)
        \end{pmatrix}}
        =
        \underset{\vec m}{\begin{pmatrix}
            x(t) \\ y(t) \\ z(t)
        \end{pmatrix}}
        \times
        \left[
            \underset{\Mat T^{-1}}{\begin{pmatrix}
            1/\Mat I_1 &0&0 \\ 0& 1/\Mat I_2 &0 \\ 0&0& 1/\Mat I_3
        \end{pmatrix}}
        \underset{\vec m}{\begin{pmatrix}
            x(t) \\ y(t) \\ z(t)
    \end{pmatrix}}
    \right]
    \end{align*}
    Rekn ut kryssproduktet:
    \begin{align}
        \label{euler_komp}
        \underset{\dot \vec m}{\begin{pmatrix}
            \dot x(t) \\ \dot y(t) \\ \dot z(t)
        \end{pmatrix}}
        =
        \underset{f(t, \vec m)}{\begin{pmatrix}
            A y(t) z(t) \\ B x(t) z(t) \\ C  x(t) y(t)
        \end{pmatrix}}
    \end{align}
    der $A$, $B$ og $C$ er konstantar
    \begin{align*}
        A = 1/I_3 - 1/I_2 \\
        B = 1/I_3 - 1/I_1 \\
        C = 1/I_2 - 1/I_1
    \end{align*}

    \subsection{Implisitt Runge-Kutta midtpunkt-metode}
    Vi ønsker å løyse differensiallikningar av sorten
    \[
        \dot y = f(t, y)
    \]
    numerisk, ved hjelp av Runge-Kutta metoden
    \begin{equation}
        y_{n + 1} = y_n + h f\left( t_n + \frac{h}{2}, \frac{1}{2}(y_n + y_{n + 1})\right) \text,
        \label{mpm}
    \end{equation}
    kalt {\em implisitt midtpunktmetode}, fordi $y_{n+1}$ avheng
    av eit estimat for $y_{n+1/2}$ (derav implisitt)
    for å estimere tangenten i midpunktet mellom $t_n$ og $t_{n+1}$.
    Substituer
    \[
        u = \frac 1 2 (y_n + y_{n + 1}) \implies y_{n+1} = 2u - y_n
    \]
    for å forenkle notasjonen litt.
    Dette gir likningssystemet
    \begin{align*}
                 & 2u - y_n = y_n + h f\left( t + h/2, u \right) \\
        \implies & u = y_n + \frac h 2 f\left( t+h/2, u \right) \\
        \implies & y_n + \frac h 2 f\left( t+h/2, u \right) - u = 0
    \end{align*}
    som må løysast med omsyn til $u$ for kvart tidssteg.
    Gitt ein verdi for $u$ reknar vi ut $y_{n+1}$:
    \begin{equation}
        y_{n+1} = y_n + h f(t + h/2, u)
    \end{equation}

    \subsection*{Anvending på problemet}
    Eit steg gjenstår før vi kan løyse problemet:
    Vi er nøtt å velgje ein måte å løyse det implisitte steget.
    Frå \eqref{euler} får vi med
    $u = (x \, y \, z)$ likninga
    \begin{equation}
        \begin{pmatrix}
            x_n \\ y_n \\ z_n
        \end{pmatrix}
        + \frac h 2
        \begin{pmatrix}
            A y z \\ B x z \\ C x y
        \end{pmatrix}
        -
        \begin{pmatrix}
            x \\ y \\ z
        \end{pmatrix}
        = 0 \text.
    \end{equation}
    Vi innlemmer $h/2$ i konstantane $A$, $B$ og $C$.
    Til slutt sit vi att med systemet
    \begin{align*}
        \left\{
        \begin{array}{c}
            x_n + \hat A y z - x = 0 \\
            y_n + \hat B x z - y = 0 \\
            z_n + \hat C x y - z = 0
        \end{array}
        \right.
    \end{align*}
    der $x_n$, $y_n$, $z_n$,
    samt $\hat A$, $\hat B$ og $\hat C$ er kjende konstantar.
    Jacobian-matrisa lar seg enkelt rekne ut ved hjelp av symbolske
    verktøy, til dømes {\tt sympy} i Python. Vi har
    \[
        \mathscr J = \begin{pmatrix}
            -1          &   \hat A z    & \hat A y \\
            \hat B z    &   -1          & \hat B x \\
            \hat C y    &   \hat C x    & -1
        \end{pmatrix} \text.
    \]
    Altso kan vi finne $(x \, y\, z)$ med å bruke newtons metode:
    \[
        u \longleftarrow u - \mathscr J^{-1} \vec F(u)
    \]
    som burde konvergere svært raskt dersom vi brukar
    førre iterasjon som startpunkt; $u_0 = (x_n \, y_n \, z_n)$.





    \printbibliography


\end{multicols*}
\end{document}
