% Document Setup
% ==============

\author{Steffen Haug}
%% remove the temporary files (.bcf, .aux, ...) if
%% you change the language, because this invalidates
%% the cached babel settings.
\usepackage[nynorsk]{babel}

\usepackage{csquotes}
\usepackage[style=phys]{biblatex}
\usepackage{scalerel}
\usepackage{relsize}

\usepackage{bm, upgreek}
\usepackage{booktabs}
\usepackage{ulem}

\usepackage{mathtools} % coloneqq
\usepackage{siunitx}
\usepackage{textcomp}

\usepackage{unicode-math}

% Get mathbb and mathcal back (unicode-math replaces them)
\DeclareMathAlphabet{\mathcal}{OMS}{cmsy}{m}{n}
\let\mathbb\relax % remove the definition by unicode-math
\DeclareMathAlphabet{\mathbb}{U}{msb}{m}{n}

\setmainfont{EB Garamond}[RawFeature={+ss06}]
\setmonofont{Courier New}
\setmathfont[StylisticSet={7, 2, 10}]{Garamond-Math}


% Custom Commands
% ---------------
\newcommand{\satan}{\rotatebox[origin=c]{180}{$\dagger$}}
\newcommand{\psatan}{\rotatebox[origin=c]{180}{$^\dagger$}}

\newcommand{\nablarule}{
    \noindent\sout{\hfill} {\footnotesize$\nabla$} \sout{\hfill}
}

%%% math
\newcommand{\Vect}[1]{{\mathbf{#1}}}
\let\vec\Vect

\newcommand{\Mat}[1]{\mathbf{\mathrm{#1}}}

\newcommand{\LRSeq}[1]{\left\{ #1 \right\}}
\newcommand{\Seq}[1]{\big\{ #1 \big\}}

\newcommand{\flatfrac}[2]{{#1}/{#2}}

%%% \argopen and \argclose creates groups without spacing,
%%% stolen from some physics package.
\NewDocumentCommand\argopen{s}{
    \IfBooleanTF{#1}{\mathopen{}\mathclose\bgroup}
                    {\mathopen{}\mathclose\bgroup\left}
}

\NewDocumentCommand\argclose{s}{
    \IfBooleanTF{#1}{\egroup}
                    {\aftergroup\egroup\right}
}

\NewDocumentCommand\D{ o g d() }{
    % Differential 'd'
    % 1 o: optional n-th derivative
    % 2 g: the (optional) thing to differentiate
    % 3 d: the (optional) thing to differentiate, but in brackets
    \IfNoValueTF{#2}
        {\IfNoValueTF{#3}
            {\mathrm d \IfNoValueTF{#1}{}{^{#1}}}
            {\mathinner{\mathrm d \IfNoValueTF{#1}{}{^{#1}}\argopen(#3\argclose)}}}
        {\mathinner{\mathrm d \IfNoValueTF{#1}{}{^{#1}}#2} \IfNoValueTF{#3}{}{(#3)}}
}

\NewDocumentCommand\DD{ s o m g d() }{{ %%% extra brackets for scoping
    % 1 s: optional * if you want a flat fraction
    % 2 o: optional n-th derivative
    % 3 m: mandatory variable to diff. by
    % 4 g: optional  variable to diff.
    % 5 d: optional  variable to diff. by in brackets
    \newcommand\Power{\IfNoValueTF{#2}{}{^{#2}}}
    \IfBooleanTF{#1}
        {\let\fractype\flatfrac}
        {\let\fractype\frac}
    \IfNoValueTF{#4}
        {\IfNoValueTF{#5}
            {\fractype{\mathrm d \Power }
                      {\mathrm d #3 \Power }}
            {\fractype{\mathrm d \Power }
                      {\mathrm d #3 \Power } \argopen(#5\argclose) }}
        {\fractype{\mathrm d \Power #4 }
                  {\mathrm d #3 \Power }}
}}

\NewDocumentCommand\PD{ s o m g d() }{{ %%% extra brackets for scoping
    % 1 s: optional * if you want a flat fraction
    % 2 o: optional n-th derivative
    % 3 m: mandatory variable to diff. by
    % 4 g: optional  variable to diff.
    % 5 d: optional  variable to diff. by in brackets
    \newcommand\Power{\IfNoValueTF{#2}{}{^{#2}}}
    \IfBooleanTF{#1}
        {\let\fractype\flatfrac}
        {\let\fractype\frac}
    \IfNoValueTF{#4}
        {\IfNoValueTF{#5}
            {\fractype{\partial \Power }
                      {\partial #3 \Power }}
            {\fractype{\partial \Power }
                      {\partial #3 \Power } \argopen(#5\argclose) }}
        {\fractype{\partial \Power #4 }
                  {\partial #3 \Power }}
}}

%% fourier transforms
%% ==================
\newcommand{\FTInt}[2]{
    \frac{1}{\sqrt{2\pi}} \smashoperator{\int\limits_{#1=-\infty\mathstrut}^{#1=\infty\mathstrut}}
    \D{#1} #2 e^{-i\omega #1}
}

\NewDocumentCommand\FT{ s o m }{
    % 1 s: optional * to typeset inverse transform
    % 2 o: optional arg to put in subscript of F
    % 3 m: mandatory. the thing we want to transform.
    \mathscr F^{\IfBooleanTF{#1}{-1}{}}
              _{\IfNoValueTF{#2}{}{#2}}
              \argopen\{ #3 \argclose\}
}

%% taylor series
\NewDocumentCommand\Tay{ o m } {
    \mathcal T_{\IfNoValueTF{#1}{}{#1}}
    \argopen\{ #2 \argclose\}
}


\newcommand{\Err}[1]{\mathrm{error}_{#1}}
\newcommand{\Diam}{\mathrm{diam}}

\newcommand{\R}{\mathbb{R}}
\newcommand{\N}{\mathbb{N}}
\newcommand{\C}{\mathbb{C}}

\newcommand{\Norm}[2][]{\left\lVert#2\right\rVert_{#1}}
\newcommand{\Jac}[1][]{\mathbf{J}_{#1}}
\newcommand{\Diag}[1]{\mathrm{diag} #1}


\newcommand{\Fig}[1]{\mbox{\bfseries\scshape figur~\ref{fig:#1}}}
\newcommand{\Tab}[1]{\mbox{\bfseries\scshape tabell~\ref{tab:#1}}}
\newcommand{\A}[1]{\mbox{\bfseries\scshape appendix~\ref{#1}}}
\newcommand{\Lst}[1]{\mbox{\bfseries\scshape listing~\ref{#1}}}

\DeclareMathOperator*{\Sub}{\Bigg\vert}

% smash bounding boxes. useful for limits etc.
% \def\mathclap#1{\text{\hbox to 0pt{\hss$\mathsurround=0pt#1$\hss}}}

% -- Text Formatting
\linespread{1.10}

% -- Custom Titles
\usepackage{titlesec}

\renewcommand\thesection{\Roman{section}}
\renewcommand\thesubsection{\normalfont\roman{subsection}}
\titleformat{name=\section,numberless}[block]{\Large\scshape}{}{0pt}{}
\titleformat{name=\section}[block]{\Large\scshape}{\thesection.}{5pt}{}

\titleformat{name=\subsection,numberless}[block]{\filcenter\large\scshape}{}{0pt}{}
\titleformat{name=\subsection}[block]{\filcenter\large\scshape}{\thesubsection.}{5pt}{}

\usepackage{moresize}
\usepackage{abstract}
\usepackage{appendix}

\renewcommand{\abstractnamefont}{\normalfont\scshape}
\renewcommand{\abstracttextfont}{\normalfont\small}

% -- Title Section
\usepackage{titling} % Customising the title section
\setlength{\droptitle}{-5\baselineskip} % Move the title up

% Document margins optimized for two-column layout; they
\usepackage[
    right=15mm,
    left=15mm,
    head=15pt,
]{geometry}
\setlength{\columnsep}{5mm}
\usepackage{afterpage}
\usepackage{pagecolor}

% -- Headers and footers
\usepackage{fancyhdr}
\renewcommand{\headrulewidth}{0pt}
\pagestyle{fancy}
\fancyhead{}
\fancyhead[C]{MA2501}
\fancyfoot{}
\fancyfoot[C]{\thepage}


% clean up the "plain" style.
\fancypagestyle{plain}{\
  \fancyhf{}
  \renewcommand{\headrulewidth}{0pt}
}


\usepackage{float}
\usepackage{multicol}
\usepackage{graphics}
\usepackage{placeins}

% -- Custom captions
\usepackage[hang,small,labelfont=bf,up,textfont=it,up]{caption}

% -- Compact lists
\usepackage{enumitem}
\setlist[itemize]{noitemsep}

% -- LISTINGS --
\usepackage{listings}
% Options for all coding environments
\lstset{
    basicstyle=\scriptsize\ttfamily
}
\lstset{
    frame=tb,
    framexleftmargin=2.5em,
    numbers=left,
    xleftmargin=2.5em,
    showstringspaces=false,
}

% The built-in python highlighter is for python2,
% so we need to modify it:
\newcommand\pythonstyle{\lstset{
language=Python,
morekeywords={self, yield, as, assert, pass},
}}

% python env.
\lstnewenvironment{python}[1][]
{
\pythonstyle
\lstset{#1}
}
{}

% inline pytohn env.
\newcommand\pythoninline[1]{{\pythonstyle\lstinline!#1!}}

\usepackage{xcolor}
\usepackage{tikz}
\usepackage{standalone}

\usepackage{caption}
\captionsetup{format = plain, font = footnotesize, labelfont = {sc,bf}}

\usepackage{multicol}
\usepackage{subcaption}
